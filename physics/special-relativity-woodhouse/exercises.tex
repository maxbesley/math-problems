\documentclass{article}

\usepackage{fancyhdr}
\usepackage{amsmath}
\usepackage{amsthm}
\usepackage{amssymb}
\usepackage{amsfonts}
\usepackage{extramarks}
\usepackage{derivative}
\usepackage{nicematrix}
\usepackage{tikz}
\usepackage[english]{babel}

%
% Basic Document Settings
%
\topmargin=-0.45in
\evensidemargin=0in
\oddsidemargin=0in
\textwidth=6.5in
\textheight=9.0in
\headsep=0.25in

\linespread{1.1}

\pagestyle{fancy}
\rhead{\textbf{\authorName}}
\chead{\textbf{Exercises for \textit{\bookTitle}}}
\cfoot{\thepage}
\lhead{\firstxmark}
\lfoot{\lastxmark}

\renewcommand\headrulewidth{0.4pt}
\renewcommand\footrulewidth{0.4pt}

\setlength\parindent{0pt}

%
% Create Exercise Sections
%
\newcommand{\enterExerciseHeader}[1]{
    \nobreak\extramarks{}{Exercise \arabic{#1} continued on next page\ldots}\nobreak{}
    \nobreak\extramarks{Exercise \arabic{#1} (continued)}{Exercise \arabic{#1} continued on next page\ldots}\nobreak{}
}

\newcommand{\exitExerciseHeader}[1]{
    \nobreak\extramarks{Exercise \arabic{#1} (continued)}{Exercise \arabic{#1} continued on next page\ldots}\nobreak{}
    \stepcounter{#1}
    \nobreak\extramarks{Exercise \arabic{#1}}{}\nobreak{}
}

\setcounter{secnumdepth}{0}
\newcounter{partCounter}
\newcounter{exerciseCounter}
\setcounter{exerciseCounter}{1}
\nobreak\extramarks{Exercise \arabic{exerciseCounter}}{}\nobreak{}

\newenvironment{exercise}[1][-1]{
    \ifnum#1>0
        \setcounter{exerciseCounter}{#1}
    \fi
    \section{Exercise \arabic{exerciseCounter}}
    \setcounter{partCounter}{1}
    \enterExerciseHeader{exerciseCounter}
}{
    \exitExerciseHeader{exerciseCounter}
}

\newcommand{\N}{\mathbb{N}}
\newcommand{\Z}{\mathbb{Z}}
\newcommand{\R}{\mathbb{R}}

\newcommand{\bookTitle}{Special Relativity}
\newcommand{\bookAuthor}{Nick Woodhouse}
\newcommand{\authorName}{Max Besley}

%
% Title Page
%
\title{
    \vspace{2in}
    \textmd{\textbf{\bookTitle:\ Exercises}}\\
    \vspace{0.1in}\large{\textbf{Textbook by \bookAuthor}}
    \vspace{3in}
}

\author{\textbf{\authorName}}
\date{}

\newcommand{\vect}[1]{\ensuremath{\mathbf{#1}}}
\newcommand{\norm}[1]{\left\lVert#1\right\rVert}
\renewcommand\qedsymbol{$\blacksquare$}

\renewcommand{\part}[1]{\textbf{\large Part \Alph{partCounter}}\stepcounter{partCounter}\\}

% Alias for the Solution section header
\newcommand{\solution}{\textbf{\large Solution}}

\begin{document}

\maketitle

\pagebreak

\section{\underline{\Large{Chapter 1}}} \vspace{5mm}

\begin{exercise}[1]
    Showing associativity is omitted. \\

    \textbf{Identity:} \\
    Set $H=I_3$ and $C=\vect{0}$ and this isometry is the identity. \\

    \textbf{Closure:} \\
    Assume $\phi(\boldsymbol{x})=H_1\vect{x}+C_1$ and $\psi(\vect{x})=H_2\vect{x}+C_2$ are isometries of $\mathbb{E}$. \\
    We have,
    \begin{equation} \tag{*}
    \begin{split}
        (\psi\circ\phi)(\vect{x}) & = \psi(\phi(\vect{x})) \\
                                  & = \psi(H_1\vect{x}+C_1) \\
                                  & = H_2(H_1\vect{x}+C_1)+C_2 \\
                                  & = H_2H_1\vect{x}+(H_2C_1+C_2)
    \end{split}
    \end{equation}
    and so the map $\psi\circ\phi$ is also an isometry of $\mathbb{E}$. \\

    \textbf{Inverses:} \\
    Let $\psi(\vect{x})=H\vect{x}+C$ be an isometry of $\mathbb{E}$. \\
    The inverse isometry map is $\psi^{-1}(\vect{x})=H^{-1}\vect{x}-H^{-1}C$.
    In particular it's easy to check that $(\psi^{-1}\circ\psi)(\vect{x})=
    \vect{x}=(\psi\circ\psi^{-1})(\vect{x})$ for any $\vect{x}\in\mathbb{E}$. \\

    Hence the isometries of Euclidean space form a group. \\

    We now show that the proper (orientation-preserving) isometries form a subgroup. \\

    The identity map is a proper isometry since $\text{det}(I_3)=0$.
    The composition of two proper isometries is a proper isometry because we will have
    $\text{det}(H_2H_1)=\text{det}(H_2)\text{det}(H_1)=1\cdot1=1$ in equation (*).
    Lastly, if $\psi(\vect{x})=H\vect{x}+C$ is a proper isometry we have $\text{det}(H^{-1})=\text{det}(H)^{-1}=1^{-1}=1$ and so the inverse map $\psi^{-1}$ is also
    a proper isometry. Thus the proper isometries of $\mathbb{E}$ form a subgroup.
\end{exercise}

\NiceMatrixOptions
{
    custom-line ={command= H, tikz= dashed, width= 1mm}, % horizontal, dashed
    custom-line = {letter= I, tikz= dashed, width= 1mm}, % vertical, dashed
}

\begin{exercise}[2]
    (i)
    We have
    \begin{gather*}
    t = t^\prime + c_0 \\
    x=v_1 t^\prime + H_{11}x^\prime + H_{12}y^\prime + H_{13}z^\prime + c_1 \\
    y=v_2 t^\prime + H_{21}x^\prime + H_{22}y^\prime + H_{23}z^\prime + c_2 \\
    z=v_3 t^\prime + H_{31}x^\prime + H_{32}y^\prime + H_{33}z^\prime + c_3
    \end{gather*}
    and by the first equation $\odv{t^\prime}{t}=1$. Using the chain rule,
    \begin{align*}
    \odv{x}{t}=\odv{x}{t^\prime}\odv{t^\prime}{t}= v_1 \cdot 1 = v_1
    \end{align*}
    and likewise $\odv{y}{t}=v_2$ and $\odv{z}{t}=v_3$. \\\\
    Thus we have $\vect{v}=\left(\odv{x}{t},\odv{y}{t},\odv{z}{t}\right)=\left(v_1,v_2,v_3\right)$
    where $\vect{v}$ is the velocity of $R^\prime$ relative to the frame of
    reference $R$. \\

    (ii)
    Suppose we have a second Galilean transformation \\
    \begin{align*}
        \begin{bmatrix} t^\prime \\ x^\prime \\ y^\prime \\ z^\prime \end{bmatrix}
        =
        \begin{bNiceMatrix}
            1 & 0 & 0 & 0 \\
            w_1 & K_{11} & K_{12} & K_{13} \\
            w_2 & K_{21} & K_{22} & K_{23} \\
            w_3 & K_{31} & K_{32} & K_{33}
        \end{bNiceMatrix}
        \begin{bmatrix} t^{\prime\prime} \\ x^{\prime\prime} \\ y^{\prime\prime} \\ z^{\prime\prime} \end{bmatrix}
        + \begin{bmatrix} d_0 \\ d_1 \\ d_2 \\ d_3 \end{bmatrix}
    \end{align*}

    Composing with the first Galilean transformation gives
    \begin{align*}
        \begin{bmatrix} t \\ x \\ y \\ z \end{bmatrix} &=
        \begin{bNiceMatrix}
            1 & 0 & 0 & 0 \\
            v_1 & H_{11} & H_{12} & H_{13} \\
            v_2 & H_{21} & H_{22} & H_{23} \\
            v_3 & H_{31} & H_{32} & H_{33}
        \end{bNiceMatrix}
        \left(
        \begin{bNiceMatrix}
            1 & 0 & 0 & 0 \\
            w_1 & K_{11} & K_{12} & K_{13} \\
            w_2 & K_{21} & K_{22} & K_{23} \\
            w_3 & K_{31} & K_{32} & K_{33}
        \end{bNiceMatrix}
        \begin{bmatrix} t^{\prime\prime} \\ x^{\prime\prime} \\ y^{\prime\prime}
        \\ z^{\prime\prime} \end{bmatrix}
        + \begin{bmatrix} d_0 \\ d_1 \\ d_2 \\ d_3 \end{bmatrix}
        \right)
        + \begin{bmatrix} c_0 \\ c_1 \\ c_2 \\ c_3 \end{bmatrix} \\[1em]
        &=
        \begin{bNiceArray}{cIccc}[margin]
            1 & 0 & 0 & 0 \\ \H
            \Block{3-1}{\vect{v}+H\vect{w}} & \Block[c]{3-3}<\large>{HK} \\
            & & & \\
            & & &
        \end{bNiceArray}
        \begin{bmatrix} t^{\prime\prime} \\ x^{\prime\prime} \\ y^{\prime\prime}
        \\ z^{\prime\prime} \end{bmatrix} +
        \begin{bNiceMatrix} d_0 \\ \Block{*-1}{d_0\vect{v}+H\vect{d}} \end{bNiceMatrix}
        + \begin{bmatrix} c_0 \\ c_1 \\ c_2 \\ c_3 \end{bmatrix}
    \end{align*}
    which again is a Galilean transformation as $HK \in SO(3)$.
    Here $\vect{v}=\begin{bmatrix} v_1 \\ v_2 \\ v_3 \end{bmatrix}$,
    $\vect{w}=\begin{bmatrix} w_1 \\ w_2 \\ w_3 \end{bmatrix}$
    and $\vect{d}=\begin{bmatrix} d_1 \\ d_2 \\ d_3 \end{bmatrix}$. \\[0.5em]

    The above vector $\vect{v}+H\vect{w}$ is consistent with the classical velocity addition
    law. Recall that the vector $\vect{w}$ comes from the inertial frame $R^\prime$, and
    so the orthogonal matrix $H$ is applied in order to translate from the coordinates
    of $R^\prime$ into the coordinates of the inertial frame $R$. \\

    The inverse of a general Galilean transformation is \\
    \begin{equation*}
        \begin{bmatrix} t^\prime \\ x^\prime \\ y^\prime \\ z^\prime \end{bmatrix}
        =
        \begin{bNiceArray}{cIccc}[margin]
            1 & 0 & 0 & 0 \\ \H
            \Block{3-1}{H^{-1}(\vect{-v})} & \Block[c]{3-3}<\large>{H^{-1}} \\
            & & & \\
            & & &
        \end{bNiceArray}
        \begin{bmatrix} t \\ x \\ y \\ z \end{bmatrix}
        + \begin{bNiceMatrix} -c_0 \\ \Block{*-1}{H^{-1}(c_0\vect{v}-\vect{c})} \end{bNiceMatrix}
    \end{equation*} \\
    which is a Galilean transformation as $H^{-1} \in SO(3)$. \\

    (iii) (A) \\
    A Galilean transformation performs the mapping $t_i \mapsto t^{\prime}_i + c_0$
    (where $i\in\{1,2\}$) on the two time coordinates. So the time separation between
    the two events in the new frame $R^\prime$ is $t^{\prime}_2+c_0-(t^{\prime}_1+c_0)
    =t^{\prime}_2-t^{\prime}_2=t_2-t_1$, using the fact that $t=t^\prime$ in Galilean relativity. \\

    (iii) (B) \\
    Assume that $t_1=t_2$. \\
    Observe that a Galilean transformation maps the coordinates $(t_1,x_1,y_1,z_1)$
    and $(t_2,x_2,y_2,z_2)$ to the new coordinates:
    \begin{align*}
        \begin{bNiceMatrix} t_1+c_0 \\ \Block{*-1}{\vect{v}t_1+H\vect{r_1}+\vect{c}} \end{bNiceMatrix}
        \hspace{1em} \text{and} \hspace{1em}
        \begin{bNiceMatrix} t_2+c_0 \\ \Block{*-1}{\vect{v}t_2+H\vect{r_2}+\vect{c}} \end{bNiceMatrix}
    \end{align*}
    where $\vect{r_i}=\begin{bmatrix} x_i \\ y_i \\ z_i \end{bmatrix}$
    and $\vect{c}=\begin{bmatrix} c_1 \\ c_2 \\ c_3 \end{bmatrix}$. \\[1em]
    Now we compute
    \begin{align*}
        \norm{\begin{bmatrix} x_2^\prime \\ y_2^\prime  \\ z_2^\prime  \end{bmatrix}-
        \begin{bmatrix} x_1^\prime \\ y_1^\prime  \\ z_1^\prime  \end{bmatrix}}
        &= \norm{\vect{v}t_2+H\vect{r_2}+\vect{c}-(\vect{v}t_1+H\vect{r_1}+\vect{c})} \\
        &= \norm{(t_2-t_1)\vect{v}+H\vect{r_2}-H\vect{r_1}} \\
        &= \norm{H\vect{r_2}-H\vect{r_1}} \\
        &= \norm{\vect{r_2}-\vect{r_1}} \\
        &= \sqrt{(x_2-x_1)^2+(y_2-y_1)^2+(z_2-z_1)^2}
    \end{align*}
    and so the distance between the two events is preserved. \\
    If the two events are not simultaneous then one of the frames may have
    moved through space during the time interval $t_2-t_1$, thus increasing or
    decreasing the measured spatial distance between the events. \\

    Here I skip the last part. Perhaps I will do it later.
\end{exercise}

\pagebreak

\section{\underline{\Large{Chapter 2}}} \vspace{5mm}

\begin{exercise}[1]
    (1) \\
    \begin{equation*}
       \begin{split}
        \text{div}(\textbf{\textit{E}}^\prime)&=\text{div }[\text{cos}(\alpha)\textbf{\textit{E}}-\text{sin}(\alpha)c\textbf{\textit{B}}] \\
        &=\text{cos}(\alpha)\text{div}(\textbf{\textit{E}})-\text{sin}(\alpha)c\ \text{div}(\textbf{\textit{B}}) \\
        &=0+0=0
        \end{split}
    \end{equation*}

    (2) \\
    \begin{equation*}
       \begin{split}
        \text{div}(\textbf{\textit{B}}^\prime)&=\text{div }[c^{-1}\text{sin}(\alpha)\textbf{\textit{E}}+\text{cos}(\alpha)\textbf{\textit{B}}] \\
        &=c^{-1}\text{sin}(\alpha)\text{div}(\textbf{\textit{E}})+\text{cos}(\alpha)\text{div}(\textbf{\textit{B}}) \\
         &=0+0=0
        \end{split}
    \end{equation*}

    (3) \\
    \begin{equation*}
       \begin{split}
        \text{curl}(\textbf{\textit{B}}^\prime)&=c^{-1}\text{sin}(\alpha)\text{curl}(\textbf{\textit{E}})+\text{cos}(\alpha)\text{curl}(\textbf{\textit{B}}) \\
        &=c^{-1}\text{sin}(\alpha)\left(-\pdv{\textbf{\textit{B}}}{t}\right)+\text{cos}(\alpha)\frac{1}{c^2}\pdv{\textbf{\textit{E}}}{t} \\
        &=\frac{1}{c^2}\text{cos}(\alpha)\pdv{\textbf{\textit{E}}}{t}-c^{-1}\text{sin}(\alpha)\pdv{\textbf{\textit{B}}}{t} \\
        &=\frac{1}{c^2}\left(\text{cos}(\alpha)\pdv{\textbf{\textit{E}}}{t}-\text{sin}(\alpha)\pdv{\left(c\textbf{\textit{B}}\right)}{t}\right) \\
        &=\frac{1}{c^2}\pdv{\textbf{\textit{E}}^\prime}{t}
        \end{split}
    \end{equation*}

    (4) \\
    \begin{equation*}
       \begin{split}
        \text{curl}(\textbf{\textit{E}}^\prime)&=\text{cos}(\alpha)\text{curl}(\textbf{\textit{E}})-\text{sin}(\alpha)c\ \text{curl}(\textbf{\textit{B}}) \\
        &=\text{cos}(\alpha)\left(-\pdv{\textbf{\textit{B}}}{t}\right)-\text{sin}(\alpha)c\ \frac{1}{c^2}\pdv{\textbf{\textit{E}}}{t} \\
        &=-\left(c^{-1}\text{sin}(\alpha)\pdv{\textbf{\textit{E}}}{t}+\text{cos}(\alpha)\pdv{\textbf{\textit{B}}}{t}\right) \\
        &=-\pdv{\textbf{\textit{B}}^\prime}{t}
        \end{split}
    \end{equation*}
\end{exercise}

\begin{exercise}[2]
    Assume $x^2+y^2<a^2$. \\
    Using an explicit formula for the curl operator we have \\
    \begin{equation*}
       \begin{split}
        \nabla\times\textbf{\textit{B}}&=\nabla\times\left(\frac{I\mu_0(-y\mathbf{\hat{i}}+x\mathbf{\hat{j}})}{2\pi a^2}\right) \\
        &=\frac{I\mu_0}{2\pi a^2}\ \nabla\times\left(-y\mathbf{\hat{i}}+x\mathbf{\hat{j}}\right) \\
        &=\frac{I\mu_0}{2\pi a^2}\ 2\mathbf{\hat{k}} \\
        &=\frac{I\mu_0}{\pi a^2}\mathbf{\hat{k}}
        \end{split}
    \end{equation*}
    Now assuming $x^2+y^2 \geq a^2$ we have
    \begin{equation*}
       \begin{split}
        \nabla\times\textbf{\textit{B}}&=\nabla\times\left(\frac{I\mu_0(-y\mathbf{\hat{i}}+x\mathbf{\hat{j}})}{2\pi (x^2+y^2)}\right) \\
        &=\frac{I\mu_0}{2\pi}\ \nabla\times\left(\frac{-y\mathbf{\hat{i}}+x\mathbf{\hat{j}}}{x^2+y^2}\right) \\
        &=\frac{I\mu_0}{2\pi a^2}\left(\frac{-x^2+y^2}{(x^2+y^2)^2}-\frac{-x^2+y^2}{(x^2+y^2)^2}\right)\mathbf{\hat{k}} \\
        &=\vect{0}
        \end{split}
    \end{equation*}
    Completing the exercise. \\
    See https://pressbooks.online.ucf.edu/osuniversityphysics2/chapter/amperes-law/ for another
    derivation of the magnetic field generated by an electric current in a straight wire.
\end{exercise}

\section{\underline{\Large{Chapter 3}}} \vspace{5mm}

\begin{exercise}[1]
\end{exercise}

\end{document}
